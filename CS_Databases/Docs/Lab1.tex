\documentclass{article}
\usepackage{xcolor} % For custom colors
\usepackage{listings}
\usepackage{graphicx}
\graphicspath{ {./images/Lab1} }

% Define SQL style
\lstdefinelanguage{SQL}{
  morekeywords={
    ALTER, ADD, DROP, SELECT, FROM, WHERE, AND, OR, INSERT, INTO, VALUES, UPDATE, SET, DELETE,
    CREATE, TABLE, PRIMARY, KEY, FOREIGN, NOT, NULL, REFERENCES, INNER, JOIN, CONSTRAINT,
    LEFT, RIGHT, FULL, ON, AS, DISTINCT, GROUP, BY, ORDER, HAVING, LIMIT, UNIQUE, CASCADE, DEFAULT,
    RENAME, DESCRIBE, TO
  },
  sensitive=false,
  morecomment=[l]{--}, % SQL single-line comments
  morecomment=[s]{/*}{*/}, % SQL multi-line comments
  morestring=[b]', % Strings in single quotes
}
\lstset{language=SQL,
    numbers=left,
    basicstyle=\ttfamily,
    showstringspaces=false,
    keywordstyle=\color{blue}\bfseries,
    commentstyle=\color{orange}\itshape,
    stringstyle=\color{green},
    frame=single,
    breaklines=true
}
    
\begin{document}
\title{CSCD 327 Lab 1}
\author{Dustin Randall}
\maketitle

\section{Create student table(\underline{ID}, name, major, city, state)}
\begin{lstlisting}
CREATE TABLE student(
  ID int,
  name varchar(20),
  major varchar(20),
  city varchar(20),
  state char(2),
  primary key(ID)
  )
\end{lstlisting}
\begin{figure}[ht]
    \centering
    \includegraphics[width=\textwidth]{create_student.png}
    \caption{Creating the student table}
\end{figure}
\pagebreak

\section{Create department table where dept\_name is unique.}
\begin{lstlisting}
CREATE TABLE department(
  dnumber int,
  dept_name varchar(20) unique,
  building varchar(20),
  primary key(dnumber)
  )
\end{lstlisting}
\begin{figure}[ht]
    \centering
    \includegraphics[width=\textwidth]{create_dept.png}
    \caption{Creating the department table}
\end{figure}
\pagebreak

\section{Split name into first and last name}
\begin{lstlisting}
ALTER TABLE student
 ADD first_name varchar(20),
 ADD last_name varchar(20),
 DROP name
\end{lstlisting}
\begin{figure}[ht]
    \centering
    \includegraphics[width=\textwidth]{alter_student.png}
    \caption{Rename the student table}
\end{figure}
\pagebreak

\section{Add foreign key to the department table's dnumber}
\begin{lstlisting}
 ALTER TABLE student
 ADD dnumber int,
 ADD constraint fk_constraint foreign key(dnumber) references department(dnumber) 
\end{lstlisting}
\begin{figure}[ht]
    \centering
    \includegraphics[width=\textwidth]{add_fk.png}
    \caption{Adding foreign key to the student table}
\end{figure}
\pagebreak

\section{Rename student table to student\_details}
\begin{lstlisting}
 RENAME TABLE student TO student_details
\end{lstlisting}
\begin{figure}[ht]
    \centering
    \includegraphics[width=\textwidth]{rename_student.png}
    \caption{Rename the student table}
\end{figure}
\pagebreak

\section{Retrieve the details of the columns of student\_details}
\begin{lstlisting}
 DESCRIBE student_details
\end{lstlisting}
\begin{figure}[ht]
    \centering
    \includegraphics[width=\textwidth]{describe_student.png}
    \caption{Describe student\_details}
\end{figure}
\pagebreak

\section{Delete both tables}
\begin{lstlisting}
 DROP TABLE student_details, department
\end{lstlisting}
\begin{figure}[ht]
    \centering
    \includegraphics[width=\textwidth]{drop_tables.png}
    \caption{Drop both tables}
\end{figure}
\end{document}
