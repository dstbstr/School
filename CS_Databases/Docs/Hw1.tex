\documentclass{article}
\usepackage[margin=1.5cm]{geometry}
\usepackage{tikz}
\usepackage[normalem]{ulem} % Dashed Underlines
\usetikzlibrary{positioning, matrix, shapes.geometric, arrows.meta}

\begin{document}
\title{CSCD 327 Homework 1}
\author{Dustin Randall}
\maketitle

\tikzset{
    er/text width/.initial=2.6cm,
    er/header fill/.initial=blue!20,
    er/double distance/.initial=1.2pt,
    entity/.style={
        matrix of nodes,
        anchor=north,
        nodes={rectangle, anchor=west, inner xsep=4pt, minimum height=2mm, text width=\pgfkeysvalueof{/tikz/er/text width}, align=left, draw=none},
        row sep=0,
        row 1/.style={nodes={fill=\pgfkeysvalueof{/tikz/er/header fill}, font=\bfseries, align=center, text width=\pgfkeysvalueof{/tikz/er/text width}, draw=black}}
    },
    DoubleLine/.style={double, double distance=\pgfkeysvalueof{/tikz/er/double distance}, line width=0.4pt},
    relationship/.style={diamond, draw, aspect=1.4, inner sep=2pt, minimum width=14mm, font=\small},
    identifying/.style={diamond, draw, DoubleLine, aspect=1.4, inner sep=2pt, minimum width=14mm},
    %arrow styles
    >=Stealth,
    arrow/.style={{-Stealth}, thick},
    arrowrev/.style={-{Stealth[reversed]}, thick}
}
\section{Draw a schema diagram for the following database.}
\begin{itemize}
\item Teams $($\underline{teamID}, team\_name, city, founded\_year, budget\_million$)$
\item Players $($\underline{playerID}, player\_name, age, position, salary\_million, goals, assists, teamID$)$
\item Coaches $($\underline{coachID}, coach\_name, age, experience\_years, teamID, salary\_million$)$
\item Team\_Stats $($\underline{teamID}, \underline{season}, wins, losses, goals\_scored, goals\_conceded$)$
\end{itemize}

\begin{tikzpicture}
  \matrix (Teams) [entity]{
    Teams\\
    \underline{teamID}\\
    team\_name\\
    city\\
    founded\_year\\
    budget\_million\\
  };

  \matrix (Players) [entity, left=18mm of Teams]{
    Players\\
    \underline{playerID}\\
    player\_name\\
    age\\
    position\\
    salary\_million\\
    goals\\
    assists\\
    teamID\\
  };

  \matrix (Coaches) [entity, right=18mm of Teams]{
    Coaches\\
    \underline{coachID}\\
    coach\_name\\
    age\\
    experience\_years\\
    teamID\\
    salary\_million\\
  };

  \matrix (TeamStats) [entity, below=12mm of Teams]{
    Team\_Stats\\
    \underline{teamID}\\
    \underline{season}\\
    wins\\
    losses\\
    goals\_scored\\
    goals\_conceded\\
  };

  % draw borders around entities
  \draw (Teams-1-1.north west) rectangle (Teams-6-1.south east);
  \draw (Players-1-1.north west) rectangle (Players-9-1.south east);
  \draw (Coaches-1-1.north west) rectangle (Coaches-7-1.south east);
  \draw (TeamStats-1-1.north west) rectangle (TeamStats-7-1.south east);

  % draw arrows for foreign keys
  \draw[arrow] (Players-9-1.east) -- ++(6mm,0) |- (Teams-2-1.west);
  \draw[arrow] (Coaches-6-1.west) -- ++(-6mm,0) |- (Teams-2-1.east);
  \draw[arrow] (TeamStats-2-1.east) -- ++(6mm,0mm) |- (Teams-2-1.east);

\end{tikzpicture}

\section{Consider the following schema: R(\underline{$A_1$}, $A_2$, $A_3$, $A_4$).}
\subsection{Is $(A_1, A_3)$ a candidate key for R? Why or why not?}
$(A_1, A_3)$ is not a candidate key for R because $A_1$ alone is a unique identifier, and candidate keys must be minimal.
\subsection{Is $(A_1, A_4)$ a superkey for R? Why or why not?}
$(A_1, A_4)$ is a superkey for R because it can uniquely identify each tuple in R. Adding attributes to a superkey produces another superkey.

\section{Draw an ER diagram for the requirements provided.}

\begin{tikzpicture}
    % Entity Sets
    \matrix(Division) [entity]{
        Division \\
        \underline{division\_name} \\
        Location \\
        \{phone\_numbers\}\\
    };
    \matrix(Staff) [entity, left=60mm of Division]{
        Staff \\
        \underline{staff\_number} \\
        Name \\
        DoB \\
        Salary\\
    };
    \matrix(Task) [entity, below=30mm of Division]{
        Task \\
        \underline{TaskId} \\
        Title \\
        Budget\\
    };
    \matrix(Dependent) [entity, below=30mm of Staff]{
        Dependent \\
        \dashuline{Name} \\
        DoB \\
        Relationship\\
    };

      % draw borders around entities
    \draw (Division-1-1.north west) rectangle (Division-4-1.south east);
    \draw (Staff-1-1.north west) rectangle (Staff-5-1.south east);
    \draw (Task-1-1.north west) rectangle (Task-4-1.south east);
    \draw[DoubleLine] (Dependent-1-1.north west) rectangle (Dependent-4-1.south east);

    % Relationship Sets
    \node[relationship, left=20mm of Staff] (Supervises) {Supervises};
    \node[relationship, left=20mm of Division, yshift=10mm] (HeadedBy) {HeadedBy};
    \node[relationship, below=5mm of HeadedBy] (AttachedTo) {AttachedTo};
    \node[relationship, below=5mm of Division] (Manages) {Manages};
    \node[identifying, above=5mm of Dependent] (DependentOf) {DependentOf};

    % draw HeadedBy attribute
    \node[above=10mm of HeadedBy] (StartDate) {start\_date};
    \draw (StartDate.north west) rectangle (StartDate.south east);

    % Arrows
    \draw[DoubleLine] (HeadedBy.east) -- (Division-1-1.west);
    \draw[arrow] (HeadedBy.west) -- (Staff-1-1.east);
    \draw[-, dashed, thick] (HeadedBy.north) -- (StartDate.south);
    \draw[arrow] (AttachedTo.east) -- (Division-4-1.west);
    \draw[DoubleLine] (AttachedTo.west) -- (Staff-5-1.east);
    \draw[-, thick] (Staff-1-1.west) -- node[midway, above, font=\scriptsize, inner sep=1pt]{Supervises} (Supervises.north);
    \draw[arrow] (Supervises.south) -- node[midway, below, font=\scriptsize, inner sep=1pt]{Supervisor} (Staff-5-1.west);

    \draw[arrow] (Manages.north) -- (Division.south);
    \draw[DoubleLine] (Manages.south) -- (Task.north);
    \draw[arrow] (DependentOf.north) -- (Staff.south);
    \draw[DoubleLine] (DependentOf.south) -- (Dependent.north);
\end{tikzpicture}
\pagebreak

\section{Create Relational Schemas for the ER diagram. Explain the creation process step by step.}
\subsection*{Convert Strong Entity Sets}
\begin{itemize}
    \item Division$($\underline{division\_name}, location$)$
    \begin{itemize}
        \item DivisionPhone$($\underline{division\_name}, \underline{phone\_number}$)$
    \end{itemize}
    \item Staff$($\underline{staff\_number}, name, DoB, salary$)$
    \item Task$($\underline{TaskId}, title, budget$)$
\end{itemize}

\subsection*{Convert Weak Entity Sets}
\begin{itemize}
    \item Dependent$($\underline{staff\_number}, \underline{name}, DoB, relationship$)$
\end{itemize}

\subsection*{Convert Relationship Sets}
\begin{itemize}
    \item HeadedBy$($\underline{division\_name}, staff\_number, start\_date$)$
    \item AttachedTo$($\underline{staff\_number}, division\_name$)$
    \item Manages$($\underline{task\_id}, division\_name$)$
    \item Supervises$($\underline{staff\_number}, supervisor\_staff\_number$)$
    \item DependentOf$($\underline{staff\_number}, \underline{name}$)$
\end{itemize}

\subsection*{Find Redundancies}
\begin{itemize}
    \item HeadedBy can be removed by adding staff\_name and start\_date to Division.
    \item AttachedTo can be removed by adding division\_name to Staff.
    \item Manages can be removed by adding division\_name to Task.
    \item DependentOf can be removed as it is the identifying relationship for Dependent.
\end{itemize}

\subsection*{Final Schemas}
\begin{itemize}
    \item Division$($\underline{division\_name}, location, headed\_by, headed\_start\_date$)$
    \item DivisionPhone$($\underline{division\_name}, \underline{phone\_number}$)$
    \item Staff$($\underline{staff\_number}, name, DoB, salary, division\_name, supervisor\_staff\_number$)$
    \item Task$($\underline{TaskId}, title, budget, division\_name$)$
    \item Dependent$($\underline{staff\_number}, \underline{name}, DoB, relationship$)$
    \item Supervises$($\underline{staff\_number}, supervisor\_staff\_number$)$
\end{itemize}

\end{document}