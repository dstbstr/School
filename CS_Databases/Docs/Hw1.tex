\documentclass{article}
\usepackage{xcolor} % For custom colors
\usepackage[margin=1cm]{geometry}
\usepackage{listings}
\usepackage{graphicx}
\usepackage{tikz}
\usetikzlibrary{positioning, matrix, shapes.geometric, arrows}

\begin{document}
\title{CSCD 327 Homework 1}
\author{Dustin Randall}
\maketitle

\tikzset{
    entity/.style={
        matrix of nodes,
        anchor=north,
        nodes={rectangle, anchor=west, inner xsep=4pt, minimum height=2mm, text width=2.5cm, align=left, draw=none},
        row sep=0,
        row 1/.style={nodes={fill=blue!20, font=\bfseries, align=center, text width=2.5cm, draw=black}}
    }
}
\section{Draw a schema diagram for the following database.}
\begin{itemize}
\item Teams $(\underline{teamID}, team\_name, city, founded\_year, budget\_million)$
\item Players $(\underline{playerID}, player\_name, age, position, salary\_million, goals, assists, teamID)$
\item Coaches $(\underline{coachID}, coach\_name, age, experience\_years, teamID, salary\_million)$
\item Team\_Stats $(\underline{teamID}, \underline{season}, wins, losses, goals\_scored, goals\_conceded)$
\end{itemize}

\begin{tikzpicture}
  \matrix (Teams) [entity]{
    Teams\\
    \underline{teamID}\\
    team\_name\\
    city\\
    founded\_year\\
    budget\_million\\
  };

  \matrix (Players) [entity, left=18mm of Teams]{
    Players\\
    \underline{playerID}\\
    player\_name\\
    age\\
    position\\
    salary\_million\\
    goals\\
    assists\\
    teamID\\
  };

  \matrix (Coaches) [entity, right=18mm of Teams]{
    Coaches\\
    \underline{coachID}\\
    coach\_name\\
    age\\
    experience\_years\\
    teamID\\
    salary\_million\\
  };

  \matrix (TeamStats) [entity, below=12mm of Teams]{
    Team\_Stats\\
    \underline{teamID}\\
    \underline{season}\\
    wins\\
    losses\\
    goals\_scored\\
    goals\_conceded\\
  };

  % draw borders around entities
  \draw (Teams-1-1.north west) rectangle (Teams-6-1.south east);
  \draw (Players-1-1.north west) rectangle (Players-9-1.south east);
  \draw (Coaches-1-1.north west) rectangle (Coaches-7-1.south east);
  \draw (TeamStats-1-1.north west) rectangle (TeamStats-7-1.south east);

  % draw arrows for foreign keys
  \draw[->, thick] (Players-9-1.east) -- ++(6mm,0) |- (Teams-2-1.west);
  \draw[->, thick] (Coaches-6-1.west) -- ++(-6mm,0) |- (Teams-2-1.east);
  \draw[->, thick] (TeamStats-2-1.east) -- ++(6mm,0mm) |- (Teams-2-1.east);

\end{tikzpicture}

\section{Consider the following schema: R(\underline{$A_1$}, $A_2$, $A_3$, $A_4$).}
\subsection{Is $(A_1, A_3)$ a candidate key for R? Why or why not?}
$(A_1, A_3)$ is not a candidate key for R because $A_1$ alone is a unique identifier, and candidate keys must be minimal.
\subsection{Is $(A_1, A_4)$ a superkey for R? Why or why not?}
$(A_1, A_4)$ is a superkey for R because it can uniquely identify each tuple in R. Adding attributes to a superkey produces another superkey.

\section{Draw an ER diagram for the requirements provided.}
\begin{itemize}
\item The organization has several divisions. Each division has a unique name, a location, and a set of phone numbers. 
Each division is headed by a staff member, and is required to store the date on which a staff member starts heading a division.
A staff member can only head 1 division, not all staff will head a division.
\item Each division manages several tasks.
A task has a unique id, a title, and a budget amount. A task is managed by exactly 1 division.
Each staff member has a unique staff number. They also have a name, a DoB, and salary.
Each staff member is attached to 1 division, and each division has multiple staff.
\item Staff member can have a supervisor, and can also supervise others.
One staff member can supervise multiple other staff members, but no staff member can have more than one supervisor.
\item Dependent information is stored for staff with dependents.
Dependents have a name, DoB, and a relationship with the staff member.
It can be assumed that no two dependents of a staff member will have the same name.
\item ER digaram should show Entity sets, and relationship sets with their attributes.
\item The cardinality constraints.
\item The participation constraints.
\item The roles of the entities in a relationship set as appropriate.
\end{itemize}

\section{Create Relational Schemas for the ER diagram. Explain the creation process step by step.}

\end{document}