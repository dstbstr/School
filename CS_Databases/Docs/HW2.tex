\documentclass{article}
\usepackage[margin=1.5cm]{geometry}
\usepackage{amssymb}
\usepackage{float}
\usepackage{caption}

\def\ojoin{\setbox0=\hbox{$\bowtie$}%
  \rule[-.02ex]{.25em}{.4pt}\llap{\rule[\ht0]{.25em}{.4pt}}}
\def\leftouterjoin{\mathbin{\ojoin\mkern-5.8mu\bowtie}}
\def\rightouterjoin{\mathbin{\bowtie\mkern-5.8mu\ojoin}}
\def\fullouterjoin{\mathbin{\ojoin\mkern-5.8mu\bowtie\mkern-5.8mu\ojoin}}

\begin{document}
\title{CSCD 327 Homework 2}
\author{Dustin Randall}
\maketitle

% student (studentID, name, major, year, departmentID)
% course (courseID, title, departmentID, credits)
% department (departmentID, dept_name, location)
% enrollment (studentID, courseID, grade)

\section{Find the names and majors of all students from the Mathematics department.}
\begin{Large}
$\Pi_{name, major}(\sigma_{dept\_name = 'Mathematics'}(student \bowtie department))$
\end{Large}

\section{For all students who have enrolled in at least one course, list the names of the
students with the titles and credits of the courses they have enrolled in.}
\begin{Large}
    $\Pi_{name, title, credits}(student \bowtie enrollment \bowtie course \bowtie_{(course.departmentID = department.departmentId)} department)$
\end{Large}

\section{List the student IDs and names for all students who have not enrolled in any course
offered by the department with department ID 101.}
\begin{Large}
    $ToRemove \leftarrow \sigma_{departmentID = 101}(student \bowtie enrollment \bowtie course)$ \\
    $\Pi_{studentID, name}(student) - \Pi_{studentID, name}(ToRemove)$
\end{Large}

\section{List all course IDs and course titles offered by the Finance department or the
Business department.}
\begin{Large}
    $\Pi_{courseID, title}(\sigma_{dept\_name = 'Finance' \lor dept\_name= 'Business'}(course \bowtie department))$
\end{Large}

\section{Find the names of students who have enrolled in both courses with course IDs MIS101 and CS202.}
\begin{Large}
    $Mis \leftarrow \sigma_{courseID = 'MIS101'}(student \bowtie enrollment)$ \\
    $Cs \leftarrow \sigma_{courseID = 'CS202'}(student \bowtie enrollment)$ \\
    $\Pi_{name}(Mis) \cap \Pi_{name}(Cs)$
\end{Large}

\section{List the course titles and their offering departments’ names for all courses that no students have enrolled in.}
\begin{Large}
    $EmptyCourses \leftarrow \Pi_{courseID}(course) - \Pi_{courseID}(enrollment)$ \\
    $\Pi_{title, dept\_name}(EmptyCourses \bowtie course \bowtie department)$
\end{Large}

\section{List the names of students who received a grade of A or B in any course, along with
the titles and names of the offering departments of those courses.}
\begin{Large}
    $GoodGrades \leftarrow \sigma_{grade = 'A' \lor grade = 'B'}(enrollment)$ \\
    $\Pi_{name, title, dept\_name}(GoodGrades \bowtie student \bowtie course \bowtie_{(course.departmentID = department.departmentId)} department)$
\end{Large}

\section{Find the names and majors of all students who have enrolled in at least one course
offered by the Physics department.}
\begin{Large}
    $PhysicsCourses \leftarrow \Pi_{courseID}(\sigma_{dept\_name = 'Physics'}(course \bowtie department))$ \\
    $PhysicsStudents \leftarrow \Pi_{studentID}(enrollment \bowtie PhysicsCourses)$ \\
    $\Pi_{name, major}(PhysicsStudents \bowtie student)$
\end{Large}

\section{List the titles of the courses that no students who are in Freshman year have enrolled
in.}

\begin{Large}
    $Freshmen \leftarrow \Pi_{studentID}(\sigma_{year = 'Freshman'}(student))$ \\
    $FreshmanCourses \leftarrow \Pi_{courseID}(enrollment \bowtie Freshmen)$ \\
    $UpperClasses \leftarrow \Pi_{courseID}(course) - FreshmanCourses$ \\
    $\Pi_{title}(UpperClasses \bowtie course)$
\end{Large}

\section{Find the names of the students who have enrolled in the course with the title Database
Systems, along with the name of the offering department.}
\begin{Large} 
    $DbCourse \leftarrow \Pi_{courseID, dept\_name}(\sigma_{title = 'Database Systems'}(course \bowtie department))$ \\
    $\Pi_{name, dept\_name}(DbCourse \bowtie enrollment \bowtie_{enrollment.studentID = student.studentID} student)$
\end{Large}

\section{Produce the output of the following queries}
% SID   name            class
% 1001  John Smith      Freshman
% 1002  Emma White      Junior
% 1003  Amber Brown     Senior
% 1004  David Hernandez Junior


% CID   club_name        SID
% C101  Robotics Club   1001
% C102  Coding Club     1002
% C103  Chess Club      1002
% C104  Science Club    1005


\begin{table}[H]{$club \leftouterjoin student$}
\begin{tabular}{|c|c|c|c|c|}
    \hline
    SID & name & class & CID & club\_name \\
    \hline
    1001 & John Smith & Freshman & C101 & Robotics Club \\
    1002 & Emma White & Junior & C102 & Coding Club \\
    1002 & Emma White & Junior & C103 & Chess Club \\
    1005 & NULL & NULL & C104 & Science Club \\
    \hline
\end{tabular} \\
A Left Outer Join of club and student will produce all rows from the club table,
and any matching rows from the student table. If there is no match, it fills in null.
There is 1 row in the club table with no match in the student table, 
so that row will have null values for the student attributes. 
The resulting table will have duplicate attributes removed, leaving us with 5 columns.\\
\end{table}

\begin{table}[H]{$club \rightouterjoin student$}
\begin{tabular}{|c|c|c|c|c|}
    \hline
    SID & name & class & CID & club\_name \\
    \hline
    1001 & John Smith & Freshman & C101 & Robotics Club \\
    1002 & Emma White & Junior & C102 & Coding Club \\
    1002 & Emma White & Junior & C103 & Chess Club \\
    1003 & Amber Brown & Senior & NULL & NULL \\
    1004 & David Hernandez & Junior & NULL & NULL \\
    \hline
\end{tabular} \\
A Right Outer Join of club and student will produce all rows from the student table,
and any matching rows from the club table. If there is no match, it fills in null.
There are 2 rows in student with no matches in club, so they appear with nulls in the club columns. \\
\end{table}

\begin{table}[H]{$student \fullouterjoin club$}
\begin{tabular}{|c|c|c|c|c|}
    \hline
    SID & name & class & CID & club\_name \\
    \hline
    1001 & John Smith & Freshman & C101 & Robotics Club \\
    1002 & Emma White & Junior & C102 & Coding Club \\
    1002 & Emma White & Junior & C103 & Chess Club \\
    1003 & Amber Brown & Senior & NULL & NULL \\
    1004 & David Hernandez & Junior & NULL & NULL \\
    1005 & NULL & NULL & C104 & Science Club \\
    \hline
\end{tabular} \\
A full outer join between student and club will produce all rows from both tables.
The matching rows will be combined, and the rest will have nulls for missing attributes.
As before, there are 5 attributes in the resulting table. \\

\end{table}

\end{document}