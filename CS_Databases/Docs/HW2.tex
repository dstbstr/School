\documentclass{article}
\usepackage[margin=1.5cm]{geometry}
\usepackage{amssymb}
%\usepackage{float}
\usepackage{caption}

\def\ojoin{\setbox0=\hbox{$\bowtie$}%
  \rule[-.02ex]{.25em}{.4pt}\llap{\rule[\ht0]{.25em}{.4pt}}}
\def\leftouterjoin{\mathbin{\ojoin\mkern-5.8mu\bowtie}}
\def\rightouterjoin{\mathbin{\bowtie\mkern-5.8mu\ojoin}}
\def\fullouterjoin{\mathbin{\ojoin\mkern-5.8mu\bowtie\mkern-5.8mu\ojoin}}

\begin{document}
\title{CSCD 327 Homework 2}
\author{Dustin Randall}
\maketitle

% student (studentID, name, major, year, departmentID)
% course (courseID, title, departmentID, credits)
% department (departmentID, dept_name, location)
% enrollment (studentID, courseID, grade)

\section{Find the names and majors of all students from the Mathematics department.}
\begin{Large}
$\Pi_{name, major}(\sigma_{dept\_name = 'Mathematics'}(student \bowtie department))$
\end{Large}

\section{For all students who have enrolled in at least one course, list the names of the
students with the titles and credits of the courses they have enrolled in.}
Need to prevent false sharing between student.departmentID and course.departmentID,
so our join for department specifies that we want the course's departmentID, not the student's. \\
\begin{Large}
    $\Pi_{name, title, credits}(student \bowtie enrollment \bowtie course \bowtie_{(course.departmentID = department.departmentId)} department)$
\end{Large}

\section{List the student IDs and names for all students who have not enrolled in any course
offered by the department with department ID 101.}
We find all of the students in courses offered by department 101,
then remove all of those from the set of all students. \\
\begin{Large}
    $ToRemove \leftarrow \sigma_{departmentID = 101}(student \bowtie enrollment \bowtie course)$ \\
    $\Pi_{studentID, name}(student) - \Pi_{studentID, name}(ToRemove)$
\end{Large}

\section{List all course IDs and course titles offered by the Finance department or the
Business department.}
\begin{Large}
    $\Pi_{courseID, title}(\sigma_{dept\_name = 'Finance' \lor dept\_name= 'Business'}(course \bowtie department))$
\end{Large}

\section{Find the names of students who have enrolled in both courses with course IDs MIS101 and CS202.}
We find each group of students, then find the intersection of those sets. \\
\begin{Large}
    $Mis \leftarrow \sigma_{courseID = 'MIS101'}(student \bowtie enrollment)$ \\
    $Cs \leftarrow \sigma_{courseID = 'CS202'}(student \bowtie enrollment)$ \\
    $\Pi_{name}(Mis) \cap \Pi_{name}(Cs)$
\end{Large}

\section{List the course titles and their offering departments’ names for all courses that no students have enrolled in.}
We could have done this using a left outer join, but it seemed easier to remove all courses from the enrollment table. \\
\begin{Large}
    $EmptyCourses \leftarrow \Pi_{courseID}(course) - \Pi_{courseID}(enrollment)$ \\
    $\Pi_{title, dept\_name}(EmptyCourses \bowtie course \bowtie department)$
\end{Large}

\section{List the names of students who received a grade of A or B in any course, along with
the titles and names of the offering departments of those courses.}
Need to prevent false sharing between student.departmentID and course.departmentID,
so our join for department specifies that we want the course's departmentID, not the student's. \\
\begin{Large}
    $GoodGrades \leftarrow \sigma_{grade = 'A' \lor grade = 'B'}(enrollment)$ \\
    $\Pi_{name, title, dept\_name}(GoodGrades \bowtie student \bowtie course \bowtie_{(course.departmentID = department.departmentId)} department)$
\end{Large}

\section{Find the names and majors of all students who have enrolled in at least one course
offered by the Physics department.}
Find all of the physics courses, then find enrollment records for those courses.
From there, we join those student ids with the student table. \\
\begin{Large}
    $PhysicsCourses \leftarrow \Pi_{courseID}(\sigma_{dept\_name = 'Physics'}(course \bowtie department))$ \\
    $PhysicsStudents \leftarrow \Pi_{studentID}(enrollment \bowtie PhysicsCourses)$ \\
    $\Pi_{name, major}(PhysicsStudents \bowtie student)$
\end{Large}

\section{List the titles of the courses that no students who are in Freshman year have enrolled
in.}
We need to remove all courses that freshmen have enrolled in from the set of all courses.
We can't simply use enrollment, because that may exclude courses where no students are enrolled. \\

\begin{Large}
    $Freshmen \leftarrow \Pi_{studentID}(\sigma_{year = 'Freshman'}(student))$ \\
    $FreshmanCourses \leftarrow \Pi_{courseID}(enrollment \bowtie Freshmen)$ \\
    $UpperClasses \leftarrow \Pi_{courseID}(course) - FreshmanCourses$ \\
    $\Pi_{title}(UpperClasses \bowtie course)$
\end{Large}

\section{Find the names of the students who have enrolled in the course with the title Database
Systems, along with the name of the offering department.}
We find the course with the specified title (presumably one, but could be multiple).
When we join with student, we need to specify that we're joining on studentID and not departmentID.
Could have done this by renaming dept\_name as another approach. \\
\begin{Large} 
    $DbCourse \leftarrow \Pi_{courseID, dept\_name}(\sigma_{title = 'Database Systems'}(course \bowtie department))$ \\
    $\Pi_{name, dept\_name}(DbCourse \bowtie enrollment \bowtie_{enrollment.studentID = student.studentID} student)$
\end{Large}

\section{Produce the output of the following queries}
% SID   name            class
% 1001  John Smith      Freshman
% 1002  Emma White      Junior
% 1003  Amber Brown     Senior
% 1004  David Hernandez Junior


% CID   club_name        SID
% C101  Robotics Club   1001
% C102  Coding Club     1002
% C103  Chess Club      1002
% C104  Science Club    1005


\begin{table*}{$club \leftouterjoin student$}
\begin{tabular}{|c|c|c|c|c|}
    \hline
    SID & name & class & CID & club\_name \\
    \hline
    1001 & John Smith & Freshman & C101 & Robotics Club \\
    1002 & Emma White & Junior & C102 & Coding Club \\
    1002 & Emma White & Junior & C103 & Chess Club \\
    1005 & NULL & NULL & C104 & Science Club \\
    \hline
\end{tabular}
\end{table*}

\begin{table*}{$club \rightouterjoin student$}
\begin{tabular}{|c|c|c|c|c|}
    \hline
    SID & name & class & CID & club\_name \\
    \hline
    1001 & John Smith & Freshman & C101 & Robotics Club \\
    1002 & Emma White & Junior & C102 & Coding Club \\
    1002 & Emma White & Junior & C103 & Chess Club \\
    1003 & Amber Brown & Senior & NULL & NULL \\
    1004 & David Hernandez & Junior & NULL & NULL \\
    \hline
\end{tabular}
\end{table*}

\begin{table*}{$student \fullouterjoin club$}
\begin{tabular}{|c|c|c|c|c|}
    \hline
    SID & name & class & CID & club\_name \\
    \hline
    1001 & John Smith & Freshman & C101 & Robotics Club \\
    1002 & Emma White & Junior & C102 & Coding Club \\
    1002 & Emma White & Junior & C103 & Chess Club \\
    1003 & Amber Brown & Senior & NULL & NULL \\
    1004 & David Hernandez & Junior & NULL & NULL \\
    1005 & NULL & NULL & C104 & Science Club \\
    \hline
\end{tabular}
\end{table*}

\end{document}