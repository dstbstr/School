\documentclass[12pt]{article}
\usepackage[margin=1.5cm]{geometry}
\usepackage{xcolor} % For custom colors
\usepackage{listings}
\usepackage{graphicx}
\graphicspath{ {./images/Lab4} }

% Define SQL style
\lstdefinelanguage{SQL}{
  morekeywords={
    ALTER, ADD, DROP, SELECT, FROM, WHERE, AND, OR, INSERT, INTO, VALUES, UPDATE, SET, DELETE,
    CREATE, TABLE, PRIMARY, KEY, FOREIGN, NOT, NULL, REFERENCES, INNER, JOIN, CONSTRAINT,
    LEFT, RIGHT, FULL, ON, AS, DISTINCT, GROUP, BY, ORDER, HAVING, LIMIT, UNIQUE, CASCADE, DEFAULT,
    RENAME, DESCRIBE, TO, IN, IF, CASE, WHEN, THEN, ELSE, END, TIMESTAMPDIFF, CURRENT_DATE, ROUND, CONCAT,
    UNION
  },
  sensitive=false,
  morecomment=[l]{--}, % SQL single-line comments
  morecomment=[s]{/*}{*/}, % SQL multi-line comments
  morestring=[b]', % Strings in single quotes
}
\definecolor{darkgreen}{rgb}{0,0.4,0}

\lstset{language=SQL,
    numbers=left,
    basicstyle=\ttfamily,
    showstringspaces=false,
    keywordstyle=\color{blue}\bfseries,
    commentstyle=\color{orange}\itshape,
    stringstyle=\color{darkgreen},
    frame=single,
    breaklines=true
}
    
\begin{document}
\title{CSCD 327 Lab 4}
\author{Dustin Randall}
\maketitle

\section{Find overweight members}
\begin{lstlisting}
SELECT 
member_id, 
bmi, 
IF(bmi > 25, 'YES', 'NO') AS overweight 
FROM (
        SELECT 
        member_id,
        ROUND(weight_kg / ((height_cm / 100) * (height_cm / 100)), 2) AS bmi
        FROM members
) AS T;
\end{lstlisting}

\begin{figure}[ht]
     \centering
     \includegraphics{Overweight.png}
     \caption{Finding overweight members}
 \end{figure}
\pagebreak

\section{Find member durations}
\begin{lstlisting}
SELECT
        member_id,
        CONCAT(first_name, ' ', last_name) AS full_name,
        TIMESTAMPDIFF(month, join_date, CURRENT_DATE()) AS duration_months
FROM members;
\end{lstlisting}

\begin{figure}[ht]
     \centering
     \includegraphics{MemberDurations.png}
     \caption{Finding member durations}
 \end{figure}
\pagebreak

\section{Find oddly specific members}
\begin{lstlisting}
SELECT member_id FROM members WHERE date_of_birth < '1996-01-01'
UNION
SELECT member_id FROM payments WHERE payment_date BETWEEN '2025-01-01' AND '2025-01-10';

\end{lstlisting}

\begin{figure}[ht]
     \centering
     \includegraphics{OddlySpecific.png}
     \caption{Finding oddly specific members}
 \end{figure}
\pagebreak

\section{Find trainer years}
\begin{lstlisting}
SELECT * FROM (
  SELECT
    CONCAT(first_name, ' ', last_name) AS full_name,
    YEAR(hire_date) as hire_year 
   FROM trainers) as T
ORDER BY full_name;
\end{lstlisting}

\begin{figure}[ht]
     \centering
     \includegraphics{TrainerYears.png}
     \caption{Finding trainer years}
 \end{figure}
\pagebreak

\section{Explain the output of the provided query}
The provided query provides the cartesian product of the members and payments tables.
Each table has 6 rows, resulting in 36 rows total.
For each row in members, every row of payments is paired with it.
This corresponds to the following in relational algebra: $members \times payments$
\pagebreak

\section{Explain the output of the provided query}
\begin{lstlisting}
SELECT *
FROM members, payments
WHERE members.member_id = payments.member_id;
\end{lstlisting}

\begin{figure}[ht]
     \centering
     \includegraphics[width=\textwidth]{AllMemberPayments.png}
     \caption{Finding member payment dates}
 \end{figure}

The provided query performs an equijoin between members and payments on the member\_id column.
This is equivalent to running the previous query, then filtering to just rows where the member\_id columns match.
This is not a natural join as the query does not deduplicate the columns.
The corresponding relational algebra is:
\begin{large}{$members \bowtie_{members.member\_id=payments.member\_id} payments$}\end{large}
\pagebreak

\section{Find member payment dates}
\begin{lstlisting}
SELECT 
  CONCAT(first_name, ' ', last_name) AS full_name,
  payment_date
FROM members, payments
WHERE members.member_id = payments.member_id;
\end{lstlisting}

\begin{figure}[ht]
     \centering
     \includegraphics{MemberPaymentDates.png}
     \caption{Finding member payment dates}
 \end{figure}
\pagebreak

\section{Find all sessions and trainer information}
\begin{lstlisting}
SELECT * FROM sessions, trainers
WHERE sessions.trainer_id = trainers.trainer_id
ORDER BY session_id
\end{lstlisting}

\begin{figure}[ht]
     \centering
     \includegraphics[width=\textwidth]{SessionsAndTrainers.png}
     \includegraphics{SessionsAndTrainers2.png}
     \caption{Finding sessions and trainers}
 \end{figure}
\pagebreak

\section{Find Mike Adams Sessions}
\begin{lstlisting}
SELECT session_id, session_date, duration_minutes 
FROM sessions, trainers
WHERE 
  sessions.trainer_id = trainers.trainer_id AND 
  (first_name = 'Mike' AND last_name = 'Adams')
ORDER BY session_id
\end{lstlisting}

\begin{figure}[ht]
     \centering
     \includegraphics{MikeAdams.png}
     \caption{Finding Mike Adams}
 \end{figure}
\pagebreak

\section{Find Sophie Lee sessions}
\begin{lstlisting}
SELECT session_id, session_date 
FROM sessions,members
WHERE first_name = 'Sophie' AND last_name = 'Lee'
\end{lstlisting}

\begin{figure}[ht]
     \centering
     \includegraphics{SophieLee.png}
     \caption{Finding Sophie Lee sessions}
 \end{figure}


\end{document}