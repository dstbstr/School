\documentclass{article}
\usepackage{mathtools}
\usepackage{listings}
\lstset{language=Java,
    numbers=left,
    basicstyle=\ttfamily,
    showstringspaces=false
}
    
\begin{document}
\title{CS320 Homework 2}
\author{Dustin Randall}
\maketitle

\section{Use at least 3 levels of a recursion tree to solve \(T(n) = 4T(\frac{n}{2}) + n^2\).}
We'll start by expanding the first few levels of the recursive tree.
\begin{align*}
    T(n) &= 4T(\frac{n}{2}) + n^2 \\
    T(\frac{n}{2}) &= 4T(\frac{n}{4}) + (\frac{n}{2})^2 \\
    T(\frac{n}{4}) &= 4T(\frac{n}{8}) + (\frac{n}{4})^2 \\
    T(\frac{n}{8}) &= 4T(\frac{n}{16}) + (\frac{n}{8})^2 \\
    \vdots \\
    T(1) &= 1
\end{align*}

Using these expansions, let's substitute back into the original equation:
\begin{align*}
    T(n) &= 4T(\frac{n}{2}) + n^2 \\
         &= 4T(\frac{n}{4}) + (\frac{n}{2})^2 + 4T(\frac{n}{4}) + (\frac{n}{2})^2 + 4T(\frac{n}{4}) + (\frac{n}{2})^2 + 4T(\frac{n}{4}) + (\frac{n}{2})^2 + n^2 \\
         &= 4T(\frac{n}{4}) + 4T(\frac{n}{4}) + 4T(\frac{n}{4}) + 4T(\frac{n}{4}) + 4(\frac{n}{2})^2 + n^2 \\
         &= 4T(\frac{n}{4}) + 4T(\frac{n}{4}) + 4T(\frac{n}{4}) + 4T(\frac{n}{4}) + 4(\frac{n}{2})^2 + n^2 \\
         &= 16T(\frac{n}{4}) + 4(\frac{n^2}{4}) + n^2 \\
         &= 16T(\frac{n}{4}) + n^2 + n^2
\end{align*}
Continuing with one more expansion
\begin{align*}
    T(n) &= 16T(\frac{n}{4}) + n^2 + n^2 \\
         &= 4T(\frac{n}{8}) + (\frac{n}{4})^2 + \dots \text(16 times) + n^2 + n^2 \\
         &= 4T(\frac{n}{8}) + \dots \text(16 times) + 16(\frac{n}{4})^2 + 2n^2 \\
         &= 64T(\frac{n}{8}) + 16(\frac{n^2}{16}) + 2n^2 \\
         &= 64T(\frac{n}{8}) + n^2 + 2n^2 \\
         &= 64T(\frac{n}{8}) + 3n^2
\end{align*}
While we could expand another level, we can see a pattern emerging, and the next level would work out to \\
\(128T(\frac{n}{16}) + 4n^2\) \\
From here, we can see that each level adds another \(n^2\) to the total, and we know that T(1) = 1.
Given that each level divides n by 2, we know that there are \(\log_2(n) + 1\) levels.
So the runtime ends up being \(\log_2(n)n^2\) or more simply \(n^2 \log(n)\).

\section{Use inductive proof to show \(T(n) = 5T(\frac{n}{4}) + n^2 = O(n^2)\).}

\section{Solve the following using the Master Theorem.}
    \subsection{\(T(n) = 25T(\frac{n}{5}) + n^2 + \log(n)\)}
    \subsection{\(T(n) = 25T(\frac{n}{5}) + 2n^3 + n \log(n)\)}
    \subsection{\(T(n) = 25T(\frac{n}{5}) + 3n^4 - 3n^2\)}
    \subsection{\(T(n) = 125T(\frac{n}{5}) + 4n^2 + 5n \log(n)\)}
    \subsection{\(T(n) = 125T(\frac{n}{5}) + 5n^3 + 2n^2\)}

\section{Given the root node of a BST, convert into a double linked list where left is the previous, and right is the next.}

\section{Compare and contrast hash tables and binary search trees.}

\end{document}