\documentclass{article}
\usepackage{mathtools}
\usepackage{float}
\usepackage[margin=2cm]{geometry}
\usepackage{listings}

\begin{document}
\section{Complexity}

\begin{table}[H]
    \begin{tabular}{|c|c|c|}
    \hline
        \textbf{Function Set} & \textbf{Meaning} & Formula\\
    \hline
        \(\Theta\) & \(==\) & \(\exists \medspace c_1 > 0, c_2 > 0, n_0 > 0 \text{ such that } c_1g(n) \leq f(n) \leq c_2g(n) \text{ for all } n \geq n_0\) \\
    \hline
        \(O\) & \(\leq\) & \(\exists \medspace c > 0, n_0 > 0 \text{ such that } 0 <= f(n) <= cg(n) \text{ for all } n \geq n_0 \) \\
    \hline
        \(\Omega\) & \(\geq\) & \(\exists \medspace c > 0, n_0 > 0 \text{ such that } 0 <= cg(n) <= f(n) \text{ for all } n \geq n_0 \) \\
    \hline
        \(o\) & \(<\) & \(\forall \medspace c > 0, \exists \medspace n_0 > 0 \text{ such that } 0 <= f(n) < cg(n) \) \\
    \hline
        \(\omega\) & \(>\) & \(\forall \medspace c > 0, \exists \medspace n_0 > 0 \text{ such that } 0 <= cg(n) < f(n) \) \\
    \hline
    \end{tabular}
\end{table}

\subsection{Example}
Generally work on the left and right side separately.
\begin{align*}
    f(n) &= 2n^2 + 4n - 3 \\
    c_1(n^2) &\leq 2n^2 + 4n - 3 \leq c_2(n^2) \\
    c_1 &\leq 2 + \frac{4}{n} - \frac{3}{n^2} \leq c_2 \text { // Divide everything by } n^2 \\
    \text{Let } n_0 = 4, c_1 = 1, c_2 = 10 \\
    1 &\leq 2 + 1 - \frac{3}{16} \leq 16 \text{ // } f(n) \text{ tends towards 2 as n increases} \\
\end{align*}

\section{Recurrence}
\subsection{Tree Method}
For each T term in the recurrence, expand it with it's next iteration. i.e. \(T(n) = 2T(n/2) + n\) the next row is \(T(n/2) = 2T(n/4) + n/2\)
We can also assume the base case where \(T(1) = 1\)

\subsection{Induction Method}
Need to prove the base case and the inductive step separately.
Let \(T(n) = 5T(\frac{n}{4}) + n^2\)
We will prove that \(T(n) = O(n^2)\) using induction.
Hint: choosing \(n_0\) to equal the denominator makes the math easier.
Let \(n_0 = 4\) and \(c = 100\) \\
\(T(4): 5T(\frac{4}{4}) + 4^2 \leq 100(4^2) => 5 + 16 \leq 1600\) \\
If we assume that \(T(k) \leq ck^2\) for all \(k \geq n_0\) we need to show that \(T(k+1) \leq c(k+1)^2\)
\begin{align*}
    T(k+1) &= 5T(\frac{k+1}{4}) + (k+1)^2 \\
           &\leq 5c(\frac{k+1}{4})^2 + (k+1)^2 \text{ // by assumption} \\
              &= 5*100(\frac{(k+1)^2}{16}) + (k+1)^2 \text{ // square top and bottom} \\
              &= \frac{500}{16}(k+1)^2 + \frac{16}{16}(k+1)^2 \text{ // distribute the 500} \\
              &= 32.25(k+1)^2 \\
            32.25(k+1)^2 &\leq 100(k+1)^2
\end{align*}

\subsection{Master Theorem}
The Master Theorem requires the form \(T(n) = aT(\frac{n}{b}) + f(n)\) where \(a \geq 1 \text{ and } b \geq 1 \text{ and both a and b are constants}\) \\
Case 3 also requires that \(af(\frac{n}{b}) \leq cf(n) \text{ for some } c < 1 \text{ and sufficiently large n}\)
\begin{table}[H]
    \begin{tabular}{|c|c|c|}
    \hline
        \textbf{Simple} & \textbf{Technical} & \textbf{Conclusion} \\
    \hline
        \(T(n) > f(n)\) & \(f(n) = O(n^{\log_b{a} - \epsilon})\) for some \(\epsilon > 0\) & \(T(n) = \Theta(n^{\log_b{a}})\) \\
    \hline
        \(T(n) == f(n)\) & \(f(n) = \Theta(n^{\log_b{a}})\) & \(T(n) = \Theta(n^{\log_b{a}} \log n)\) \\
    \hline
        \(T(n) < f(n)\) & \(f(n) = \Omega(n^{\log_b{a} + \epsilon})\) for some \(\epsilon > 0\) & \(T(n) = \Theta(f(n))\) \\
    \hline
    \end{tabular}
\end{table}

\section{RBT}
\subsection{Insert}
\begin{lstlisting}
Tree.Insert(z)
z.Color = red
while(z.Parent && z.Parent.Color == red)
    if(z.Uncle.Color == red) 
        z.Parent.Color = black; z.Uncle.Color = black; z.Grandparent.Color = red; 
        z = z.Grandparent;
    else if(z == z.Parent.Right && z.Parent == z.Grandparent.Left) 
        RotateLeft(z.Parent); z = z.Left;
    else if(z == z.Parent.Left && z.Parent == z.Grandparent.Right) 
        RotateRight(z.Parent); z = z.Right;
    else if(z == z.Parent.Left && z.Parent == z.Grandparent.Left) 
        z.Parent.Color = black; z.Grandparent.Color = red; 
        RotateRight(z.Grandparent);
    else if(z == z.Parent.Right && z.Parent == z.Grandparent.Right) 
        z.Parent.Color = black; z.Grandparent.Color = red; 
        RotateLeft(z.Grandparent);
Tree.Root.Color = black;
\end{lstlisting}

\section{Math rules}
\begin{table}[H]
    \begin{tabular}{|c|c|}
    \hline
        \textbf{From} & \textbf{To} \\
    \hline
        \(\log_b{(xy)}\) & \(\log_b{(x)} + \log_b{(y)}\) \\
    \hline
        \(\log_b{(\frac{x}{y})}\) & \(\log_b{(x)} - \log_b{(y)}\) \\
    \hline
        \(\log_b{(x^k)}\) & \(k \log_b{(x)}\) \\
    \hline
        \(\sqrt{xy}\) & \(\sqrt{x} \cdot \sqrt{y}\) \\
    \hline
        \(\sqrt{\frac{x}{y}}\) & \(\frac{\sqrt{x}}{\sqrt{y}}\) \\
    \hline
        \(\sqrt{x^k}\) & \(x^{\frac{k}{2}}\) \\
    \hline
        \(x^{ab}\) & \((x^a)^b\) \\
    \hline
        \(x^{a+b}\) & \(x^a \cdot x^b\) \\
    \hline
        \((\frac{a}{b})^2\) & \(\frac{a^2}{b^2}\) \\
    \hline
    \end{tabular}
\end{table}

\end{document}